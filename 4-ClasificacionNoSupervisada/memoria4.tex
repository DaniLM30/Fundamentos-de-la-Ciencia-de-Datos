\documentclass [a4paper] {article}

\usepackage[utf8]{inputenc}

\title{Practica 4: Fundamentos de la Ciencia de Datos}
\author{
  Daniel Lopez Moreno\\
  \and
  Alejandro Fernandez Maceira\\
  \and
  Alvaro Maestre Santa
}

\usepackage{Sweave}
\begin{document}

\maketitle

\section{Clasificación no supervisada}
En este apartado vamos a realizar una clasificación no supervisada con unos datos proporcionados
por el profesor. Para ello usaremos el algoritmo K-means, sobre la muestra utilizada en clase, a 
parte se deberá obtener \textbf{los centroides de los clusters obtenidos}.\\
En primer lugar, debemos tener la librería \textbf{stats}, si no la tenemos, tenemos que instalarla.\\

Ahora procedemos a cargar los datos en una matriz para posteriormente hacer la transpuesta y hacer el
algoritmo \textbf{K-means}, como ya he mencionado antes, los datos que cargamos en la matriz, están proporcionados
por el profesor:

\begin{Schunk}
\begin{Sinput}
> (m<-matrix(c(4,4,3,5,1,2,5,5,0,1,2,2,4,5,2,1),2,8))
\end{Sinput}
\begin{Soutput}
     [,1] [,2] [,3] [,4] [,5] [,6] [,7] [,8]
[1,]    4    3    1    5    0    2    4    2
[2,]    4    5    2    5    1    2    5    1
\end{Soutput}
\end{Schunk}

Ahora calculamos la transpuesta de la matriz:

\begin{Schunk}
\begin{Sinput}
> (m<-t(m))
\end{Sinput}
\begin{Soutput}
     [,1] [,2]
[1,]    4    4
[2,]    3    5
[3,]    1    2
[4,]    5    5
[5,]    0    1
[6,]    2    2
[7,]    4    5
[8,]    2    1
\end{Soutput}
\end{Schunk}

A continuación, cargamos otra matriz de datos, y posteriormente haremos la transpuesta:

\begin{Schunk}
\begin{Sinput}
> (c<-matrix(c(0,1,2,2),2,2))
\end{Sinput}
\begin{Soutput}
     [,1] [,2]
[1,]    0    2
[2,]    1    2
\end{Soutput}
\end{Schunk}

Como ya hemos mencionado, ahora procedemos a calcular la transpuesta de la matriz cargada posteriormente:

\begin{Schunk}
\begin{Sinput}
> (c<-t(c))
\end{Sinput}
\begin{Soutput}
     [,1] [,2]
[1,]    0    1
[2,]    2    2
\end{Soutput}
\end{Schunk}

Ahora procedemos a realizar el algoritmo \textbf{K-means} para la clasificación, en este comando, tenemos que meter
las dos transpuestas calculadas previamente y posteriormente ponemos el número de iteraciones que queremos
que haga \textbf{K-means}, en este caso 4, cuando llegue a la 4 iteración parará.
El resultado obtenido es el siguiente:

\begin{Schunk}
\begin{Sinput}
> (clasificaciones<-kmeans(m,c,4))
\end{Sinput}
\begin{Soutput}
K-means clustering with 2 clusters of sizes 4, 4

Cluster means:
  [,1] [,2]
1 1.25 1.50
2 4.00 4.75

Clustering vector:
[1] 2 2 1 2 1 1 2 1

Within cluster sum of squares by cluster:
[1] 3.75 2.75
 (between_SS / total_SS =  84.8 %)

Available components:

[1] "cluster"      "centers"      "totss"        "withinss"     "tot.withinss"
[6] "betweenss"    "size"         "iter"         "ifault"      
\end{Soutput}
\end{Schunk}

Ahora vamos a proceder a explicar los resultados obtenidos:
\begin{itemize}
	\item \textbf{En primer lugar}: tenemos dos clusters de 4 y 4 datos cada uno para un total de 8 registros.
	\item \textbf{cluster means}: en este apartado, el algoritmo calcula la medida más óptima para hallar
	los centroides, los cuales, muestra en \textbf{Cluster Means}, cuando detecta estos, en este caso,
	detecta 2 centroides. 
	\item \textbf{Clustering vector}: indica el pronóstico para cada registro testeado con el algoritmo.
	\item \textbf{[1] 3.75 2.75}: estos valores, es la inercia intra-clúster de cada grupo.
	\item \textbf{between\_SS / total\_SS}: es una medida de calidad e indica que tanto están separados los
	grupos de manera inter-cluster en relación al agrupamiento intra-cluster, mientras, se esté más cercano
	al 100%, mayor será la calidad del modelo.
	\item \textbf{Available components}: tenemos los elementos disponibles del modelo, los cuales, vamos a explicar a continuación:
	\begin{itemize}
		\item \textbf{Cluster}: La categorización asignada a cada observación de los datos introducidos o dataset
		en función a su cercanía a estos centros.
		\item \textbf{Centers}: Los centroides.
		\item \textbf{Totss}: Inercia total del conjunto de datos.
		\item \textbf{Withinss}: Inercia intra-clases de cada uno de los grupos.
		\item \textbf{Tot.withinss}: Inercia intra-clases total.
		\item \textbf{Betweenss}: Inercia inter-clases.
		\item \textbf{Size}: El tamaño de cada grupo.
		\item \textbf{Iter}: Número de iteraciones empleado.
	\end{itemize}
\end{itemize}

Ahora unimos, con la función \textbf{cbind}, a la matriz de datos(clasifiaciones), en este caso, une el vector
con la matriz m:

\begin{Schunk}
\begin{Sinput}
> (m = cbind(clasificaciones$cluster,m))
\end{Sinput}
\begin{Soutput}
     [,1] [,2] [,3]
[1,]    2    4    4
[2,]    2    3    5
[3,]    1    1    2
[4,]    2    5    5
[5,]    1    0    1
[6,]    1    2    2
[7,]    2    4    5
[8,]    1    2    1
\end{Soutput}
\end{Schunk}

Ahora con la funcion \textbf{subset}, cogemos de la columna 1, aquellos valores que sean igual a 1, y creamos 
una matriz con las filas cuyo valor de la columna 1 es 1:

\begin{Schunk}
\begin{Sinput}
> (mc1 = subset(m,m[,1] == 1))
\end{Sinput}
\begin{Soutput}
     [,1] [,2] [,3]
[1,]    1    1    2
[2,]    1    0    1
[3,]    1    2    2
[4,]    1    2    1
\end{Soutput}
\end{Schunk}

Hacemos lo mismo con la columna 1, pero que contengan el valor 2:

\begin{Schunk}
\begin{Sinput}
> (mc2 = subset(m,m[,1] == 2))
\end{Sinput}
\begin{Soutput}
     [,1] [,2] [,3]
[1,]    2    4    4
[2,]    2    3    5
[3,]    2    5    5
[4,]    2    4    5
\end{Soutput}
\end{Schunk}

Quitamos la primera columna de las dos matrices generadas anteriormente, quedando los siguientes resultados:

\begin{Schunk}
\begin{Sinput}
> (mc1 = mc1[,-1])
\end{Sinput}
\begin{Soutput}
     [,1] [,2]
[1,]    1    2
[2,]    0    1
[3,]    2    2
[4,]    2    1
\end{Soutput}
\begin{Sinput}
> (mc2 = mc2[,-1])
\end{Sinput}
\begin{Soutput}
     [,1] [,2]
[1,]    4    4
[2,]    3    5
[3,]    5    5
[4,]    4    5
\end{Soutput}
\end{Schunk}

Estas dos matrices, son las matrices de los centroides resultantes.

\section{Desarrollo por parte del grupo}

En este apartado vamos a realizar una \textbf{Clasificación no supervisada} sobre una base de datos que reune información
de los personajes de la saga de películas \textbf{Star Wars}.
Este archivo \textbf{"characters.csv"} está compuesto 87 personajes con atributos repartidos en 10 columnas:
\begin{itemize}
	\item \textbf{Name}: (Nombre)
	\item \textbf{Height}: (Altura) 
	\item \textbf{Mass}: (Peso)
	\item \textbf{Hair\_color}: (Color de Pelo)
	\item \textbf{Skin\_color}: (Color de Piel)
	\item \textbf{Eye\_color}: (Color de Ojos)
	\item \textbf{Birth\_year}: (Año de nacimiento)
	\item \textbf{Gender}: (Genero)
	\item \textbf{Homeworld}: (Hogar)
	\item \textbf{Species}: (Especie)
\end{itemize}

Nuestro análisis de clasificación se centrará en \textit{clusterizar} los personajes según su \textbf{altura y peso}.
Para empezar leeremos nuestro archivo .csv y lo almacenaremos en nuestra variable \textbf{personajes}:

\begin{Schunk}
\begin{Sinput}
> personajes<- read.csv("characters.csv")
\end{Sinput}
\end{Schunk}

Algunos de nuestros personajes no tienen la suficiente información sobre su altura o peso como para clasificarlos, y en
estos atributos cuentan con el valor \textbf{NA}. Como no tenemos información sobre estos personajes, no podemos clasificarlos,
asi que debemos eliminarlos de nuestra matriz utilizando la funcion \textbf{complete.cases} y especificando que queremos trabajar
sobre las columnas de altura y peso (columnas 2 y 3), el resto no nos importa que tengan NA:

\begin{Schunk}
\begin{Sinput}
> (personajes <- personajes[complete.cases(personajes[, 2:3]),])
\end{Sinput}
\begin{Soutput}
                    name height  mass    hair_color          skin_color
1         Luke Skywalker    172    77         blond                fair
2                  C-3PO    167    75          <NA>                gold
3                  R2-D2     96    32          <NA>         white, blue
4            Darth Vader    202   136          none               white
5            Leia Organa    150    49         brown               light
6              Owen Lars    178   120   brown, grey               light
7     Beru Whitesun lars    165    75         brown               light
8                  R5-D4     97    32          <NA>          white, red
9      Biggs Darklighter    183    84         black               light
10        Obi-Wan Kenobi    182    77 auburn, white                fair
11      Anakin Skywalker    188    84         blond                fair
13             Chewbacca    228   112         brown                <NA>
14              Han Solo    180    80         brown                fair
15                Greedo    173    74          <NA>               green
16 Jabba Desilijic Tiure    175 1,358          <NA>    green-tan, brown
17        Wedge Antilles    170    77         brown                fair
18      Jek Tono Porkins    180   110         brown                fair
19                  Yoda     66    17         white               green
20             Palpatine    170    75          grey                pale
21             Boba Fett    183  78.2         black                fair
22                 IG-88    200   140          none               metal
23                 Bossk    190   113          none               green
24      Lando Calrissian    177    79         black                dark
25                 Lobot    175    79          none               light
26                Ackbar    180    83          none        brown mottle
29 Wicket Systri Warrick     88    20         brown               brown
30             Nien Nunb    160    68          none                grey
31          Qui-Gon Jinn    193    89         brown                fair
32           Nute Gunray    191    90          none       mottled green
34         Jar Jar Binks    196    66          none              orange
35          Roos Tarpals    224    82          none                grey
39               Sebulba    112    40          none           grey, red
42            Darth Maul    175    80          none                 red
44           Ayla Secura    178    55          none                blue
45              Dud Bolt     94    45          none          blue, grey
47        Ben Quadinaros    163    65          none grey, green, yellow
48            Mace Windu    188    84          none                dark
49          Ki-Adi-Mundi    198    82         white                pale
50             Kit Fisto    196    87          none               green
52            Adi Gallia    184    50          none                dark
55              Plo Koon    188    80          none              orange
57          Gregar Typho    185    85         black                dark
60     Poggle the Lesser    183    80          none               green
61       Luminara Unduli    170  56.2         black              yellow
62         Barriss Offee    166    50         black              yellow
64                 Dooku    193    80         white                fair
66            Jango Fett    183    79         black                 tan
67            Zam Wesell    168    55        blonde fair, green, yellow
68       Dexter Jettster    198   102          none               brown
69               Lama Su    229    88          none                grey
72         Ratts Tyerell     79    15          none          grey, blue
74            Wat Tambor    193    48          none         green, grey
76              Shaak Ti    178    57          none    red, blue, white
77              Grievous    216   159          none        brown, white
78               Tarfful    234   136         brown               brown
79       Raymus Antilles    188    79         brown               light
80             Sly Moore    178    48          none                pale
81            Tion Medon    206    80          none                grey
87        Padmé Amidala    165    45         brown               light
       eye_color birth_year        gender      homeworld        species
1           blue      19BBY          male       Tatooine          Human
2         yellow     112BBY          <NA>       Tatooine          Droid
3            red      33BBY          <NA>          Naboo          Droid
4         yellow    41.9BBY          male       Tatooine          Human
5          brown      19BBY        female       Alderaan          Human
6           blue      52BBY          male       Tatooine          Human
7           blue      47BBY        female       Tatooine          Human
8            red       <NA>          <NA>       Tatooine          Droid
9          brown      24BBY          male       Tatooine          Human
10     blue-gray      57BBY          male        Stewjon          Human
11          blue    41.9BBY          male       Tatooine          Human
13          blue     200BBY          male       Kashyyyk        Wookiee
14         brown      29BBY          male       Corellia          Human
15         black      44BBY          male          Rodia         Rodian
16        orange     600BBY hermaphrodite      Nal Hutta           Hutt
17         hazel      21BBY          male       Corellia          Human
18          blue       <NA>          male     Bestine IV          Human
19         brown     896BBY          male           <NA> Yoda's species
20        yellow      82BBY          male          Naboo          Human
21         brown    31.5BBY          male         Kamino          Human
22           red      15BBY          none           <NA>          Droid
23           red      53BBY          male      Trandosha     Trandoshan
24         brown      31BBY          male        Socorro          Human
25          blue      37BBY          male         Bespin          Human
26        orange      41BBY          male       Mon Cala   Mon Calamari
29         brown       8BBY          male          Endor           Ewok
30         black       <NA>          male        Sullust      Sullustan
31          blue      92BBY          male           <NA>          Human
32           red       <NA>          male Cato Neimoidia      Neimodian
34        orange      52BBY          male          Naboo         Gungan
35        orange       <NA>          male          Naboo         Gungan
39        orange       <NA>          male      Malastare            Dug
42        yellow      54BBY          male       Dathomir         Zabrak
44         hazel      48BBY        female         Ryloth        Twi'lek
45        yellow       <NA>          male        Vulpter     Vulptereen
47        orange       <NA>          male           Tund          Toong
48         brown      72BBY          male     Haruun Kal          Human
49        yellow      92BBY          male          Cerea         Cerean
50         black       <NA>          male    Glee Anselm       Nautolan
52          blue       <NA>        female      Coruscant     Tholothian
55         black      22BBY          male          Dorin        Kel Dor
57         brown       <NA>          male          Naboo          Human
60        yellow       <NA>          male       Geonosis      Geonosian
61          blue      58BBY        female         Mirial       Mirialan
62          blue      40BBY        female         Mirial       Mirialan
64         brown     102BBY          male        Serenno          Human
66         brown      66BBY          male   Concord Dawn          Human
67        yellow       <NA>        female          Zolan       Clawdite
68        yellow       <NA>          male           Ojom       Besalisk
69         black       <NA>          male         Kamino       Kaminoan
72          <NA>       <NA>          male    Aleen Minor         Aleena
74          <NA>       <NA>          male          Skako        Skakoan
76         black       <NA>        female          Shili        Togruta
77 green, yellow       <NA>          male          Kalee        Kaleesh
78          blue       <NA>          male       Kashyyyk        Wookiee
79         brown       <NA>          male       Alderaan          Human
80         white       <NA>        female         Umbara           <NA>
81         black       <NA>          male         Utapau         Pau'an
87         brown      46BBY        female          Naboo          Human
\end{Soutput}
\end{Schunk}

Ahora que ya tenemos nuestra matriz con los datos es hora de trabajar con ella. Crearemos una matriz a parte para aislar las 
medidas de altura y peso:

\begin{Schunk}
\begin{Sinput}
> height<-personajes$height
> mass<-personajes$mass
> (medidas<-cbind(height,mass))
\end{Sinput}
\begin{Soutput}
      height mass
 [1,]    172   27
 [2,]    167   26
 [3,]     96   13
 [4,]    202    7
 [5,]    150   17
 [6,]    178    6
 [7,]    165   26
 [8,]     97   13
 [9,]    183   33
[10,]    182   27
[11,]    188   33
[12,]    228    4
[13,]    180   30
[14,]    173   25
[15,]    175    1
[16,]    170   27
[17,]    180    3
[18,]     66   11
[19,]    170   26
[20,]    183   28
[21,]    200    8
[22,]    190    5
[23,]    177   29
[24,]    175   29
[25,]    180   32
[26,]     88   12
[27,]    160   24
[28,]    193   37
[29,]    191   38
[30,]    196   23
[31,]    224   31
[32,]    112   14
[33,]    175   30
[34,]    178   19
[35,]     94   15
[36,]    163   22
[37,]    188   33
[38,]    198   31
[39,]    196   35
[40,]    184   18
[41,]    188   30
[42,]    185   34
[43,]    183   30
[44,]    170   20
[45,]    166   18
[46,]    193   30
[47,]    183   29
[48,]    168   19
[49,]    198    2
[50,]    229   36
[51,]     79    9
[52,]    193   16
[53,]    178   21
[54,]    216   10
[55,]    234    7
[56,]    188   29
[57,]    178   16
[58,]    206   30
[59,]    165   15
\end{Soutput}
\end{Schunk}

\subsection{Clasificacion con k-means}

Crearemos los centroides de nuestros clusters. Para diferenciarnos del ejercicio realizado en clase realizaremos la clasificación
k-means separando en 3 clusters en vez de en 2. Los clusters empezarán siendo \textbf{C1(200cm, 100kg)}, \textbf{C2(150cm, 75kg)} y
 \textbf{C3(100cm, 50kg)}:

\begin{Schunk}
\begin{Sinput}
> centroides <-matrix(c(200,100,150,75,100,50),2,3)
> (centroides <- t(centroides))
\end{Sinput}
\begin{Soutput}
     [,1] [,2]
[1,]  200  100
[2,]  150   75
[3,]  100   50
\end{Soutput}
\end{Schunk}

Una vez creados los centroides y aisladas nuestras medidas vamos a realizar la clasificación del mismo modo que la hemos realizado 
en clase, y posteriormente \textbf{utilizaremos otros métodos de clasificación para comparar} los resultados:

\begin{Schunk}
\begin{Sinput}
> (clasiKmeans<-kmeans(medidas,centroides,6))
\end{Sinput}
\begin{Soutput}
K-means clustering with 3 clusters of sizes 9, 43, 7

Cluster means:
     height     mass
1 215.22222 15.00000
2 179.02326 24.34884
3  90.28571 12.42857

Clustering vector:
 [1] 2 2 3 1 2 2 2 3 2 2 2 1 2 2 2 2 2 3 2 2 1 2 2 2 2 3 2 2 2 2 1 3 2 2 3 2 2 2
[39] 2 2 2 2 2 2 2 2 2 2 1 1 3 2 2 1 1 2 2 1 2

Within cluster sum of squares by cluster:
[1] 2985.556 8258.744 1309.143
 (between_SS / total_SS =  84.2 %)

Available components:

[1] "cluster"      "centers"      "totss"        "withinss"     "tot.withinss"
[6] "betweenss"    "size"         "iter"         "ifault"      
\end{Soutput}
\end{Schunk}

Hemos observado que los centroides se han desplazado hacia los valores \textbf{C1(215.22222, 15.00000)}, \textbf{C2(179.02326, 24.34884)}
y \textbf{C3(90.28571 12.42857)}.
Después de clasificar mediante k-means, añadimos la columna de los clusters a la matriz de personajes y los separamos segun su cluster:

\begin{Schunk}
\begin{Sinput}
> personajesKmeans = cbind(clasiKmeans$cluster,personajes)
\end{Sinput}
\end{Schunk}

\textbf{Cluster 1:}

\begin{Schunk}
\begin{Sinput}
> (persc1 = subset(personajesKmeans,personajesKmeans[,1] == 1))
\end{Sinput}
\begin{Soutput}
   clasiKmeans$cluster            name height mass hair_color   skin_color
4                    1     Darth Vader    202  136       none        white
13                   1       Chewbacca    228  112      brown         <NA>
22                   1           IG-88    200  140       none        metal
35                   1    Roos Tarpals    224   82       none         grey
68                   1 Dexter Jettster    198  102       none        brown
69                   1         Lama Su    229   88       none         grey
77                   1        Grievous    216  159       none brown, white
78                   1         Tarfful    234  136      brown        brown
81                   1      Tion Medon    206   80       none         grey
       eye_color birth_year gender homeworld  species
4         yellow    41.9BBY   male  Tatooine    Human
13          blue     200BBY   male  Kashyyyk  Wookiee
22           red      15BBY   none      <NA>    Droid
35        orange       <NA>   male     Naboo   Gungan
68        yellow       <NA>   male      Ojom Besalisk
69         black       <NA>   male    Kamino Kaminoan
77 green, yellow       <NA>   male     Kalee  Kaleesh
78          blue       <NA>   male  Kashyyyk  Wookiee
81         black       <NA>   male    Utapau   Pau'an
\end{Soutput}
\end{Schunk}

\textbf{Cluster 2:}

\begin{Schunk}
\begin{Sinput}
> (persc2 = subset(personajesKmeans,personajesKmeans[,1] == 2))
\end{Sinput}
\begin{Soutput}
   clasiKmeans$cluster                  name height  mass    hair_color
1                    2        Luke Skywalker    172    77         blond
2                    2                 C-3PO    167    75          <NA>
5                    2           Leia Organa    150    49         brown
6                    2             Owen Lars    178   120   brown, grey
7                    2    Beru Whitesun lars    165    75         brown
9                    2     Biggs Darklighter    183    84         black
10                   2        Obi-Wan Kenobi    182    77 auburn, white
11                   2      Anakin Skywalker    188    84         blond
14                   2              Han Solo    180    80         brown
15                   2                Greedo    173    74          <NA>
16                   2 Jabba Desilijic Tiure    175 1,358          <NA>
17                   2        Wedge Antilles    170    77         brown
18                   2      Jek Tono Porkins    180   110         brown
20                   2             Palpatine    170    75          grey
21                   2             Boba Fett    183  78.2         black
23                   2                 Bossk    190   113          none
24                   2      Lando Calrissian    177    79         black
25                   2                 Lobot    175    79          none
26                   2                Ackbar    180    83          none
30                   2             Nien Nunb    160    68          none
31                   2          Qui-Gon Jinn    193    89         brown
32                   2           Nute Gunray    191    90          none
34                   2         Jar Jar Binks    196    66          none
42                   2            Darth Maul    175    80          none
44                   2           Ayla Secura    178    55          none
47                   2        Ben Quadinaros    163    65          none
48                   2            Mace Windu    188    84          none
49                   2          Ki-Adi-Mundi    198    82         white
50                   2             Kit Fisto    196    87          none
52                   2            Adi Gallia    184    50          none
55                   2              Plo Koon    188    80          none
57                   2          Gregar Typho    185    85         black
60                   2     Poggle the Lesser    183    80          none
61                   2       Luminara Unduli    170  56.2         black
62                   2         Barriss Offee    166    50         black
64                   2                 Dooku    193    80         white
66                   2            Jango Fett    183    79         black
67                   2            Zam Wesell    168    55        blonde
74                   2            Wat Tambor    193    48          none
76                   2              Shaak Ti    178    57          none
79                   2       Raymus Antilles    188    79         brown
80                   2             Sly Moore    178    48          none
87                   2        Padmé Amidala    165    45         brown
            skin_color eye_color birth_year        gender      homeworld
1                 fair      blue      19BBY          male       Tatooine
2                 gold    yellow     112BBY          <NA>       Tatooine
5                light     brown      19BBY        female       Alderaan
6                light      blue      52BBY          male       Tatooine
7                light      blue      47BBY        female       Tatooine
9                light     brown      24BBY          male       Tatooine
10                fair blue-gray      57BBY          male        Stewjon
11                fair      blue    41.9BBY          male       Tatooine
14                fair     brown      29BBY          male       Corellia
15               green     black      44BBY          male          Rodia
16    green-tan, brown    orange     600BBY hermaphrodite      Nal Hutta
17                fair     hazel      21BBY          male       Corellia
18                fair      blue       <NA>          male     Bestine IV
20                pale    yellow      82BBY          male          Naboo
21                fair     brown    31.5BBY          male         Kamino
23               green       red      53BBY          male      Trandosha
24                dark     brown      31BBY          male        Socorro
25               light      blue      37BBY          male         Bespin
26        brown mottle    orange      41BBY          male       Mon Cala
30                grey     black       <NA>          male        Sullust
31                fair      blue      92BBY          male           <NA>
32       mottled green       red       <NA>          male Cato Neimoidia
34              orange    orange      52BBY          male          Naboo
42                 red    yellow      54BBY          male       Dathomir
44                blue     hazel      48BBY        female         Ryloth
47 grey, green, yellow    orange       <NA>          male           Tund
48                dark     brown      72BBY          male     Haruun Kal
49                pale    yellow      92BBY          male          Cerea
50               green     black       <NA>          male    Glee Anselm
52                dark      blue       <NA>        female      Coruscant
55              orange     black      22BBY          male          Dorin
57                dark     brown       <NA>          male          Naboo
60               green    yellow       <NA>          male       Geonosis
61              yellow      blue      58BBY        female         Mirial
62              yellow      blue      40BBY        female         Mirial
64                fair     brown     102BBY          male        Serenno
66                 tan     brown      66BBY          male   Concord Dawn
67 fair, green, yellow    yellow       <NA>        female          Zolan
74         green, grey      <NA>       <NA>          male          Skako
76    red, blue, white     black       <NA>        female          Shili
79               light     brown       <NA>          male       Alderaan
80                pale     white       <NA>        female         Umbara
87               light     brown      46BBY        female          Naboo
        species
1         Human
2         Droid
5         Human
6         Human
7         Human
9         Human
10        Human
11        Human
14        Human
15       Rodian
16         Hutt
17        Human
18        Human
20        Human
21        Human
23   Trandoshan
24        Human
25        Human
26 Mon Calamari
30    Sullustan
31        Human
32    Neimodian
34       Gungan
42       Zabrak
44      Twi'lek
47        Toong
48        Human
49       Cerean
50     Nautolan
52   Tholothian
55      Kel Dor
57        Human
60    Geonosian
61     Mirialan
62     Mirialan
64        Human
66        Human
67     Clawdite
74      Skakoan
76      Togruta
79        Human
80         <NA>
87        Human
\end{Soutput}
\end{Schunk}

\textbf{Cluster 3:}

\begin{Schunk}
\begin{Sinput}
> (persc3 = subset(personajesKmeans,personajesKmeans[,1] == 3))
\end{Sinput}
\begin{Soutput}
   clasiKmeans$cluster                  name height mass hair_color  skin_color
3                    3                 R2-D2     96   32       <NA> white, blue
8                    3                 R5-D4     97   32       <NA>  white, red
19                   3                  Yoda     66   17      white       green
29                   3 Wicket Systri Warrick     88   20      brown       brown
39                   3               Sebulba    112   40       none   grey, red
45                   3              Dud Bolt     94   45       none  blue, grey
72                   3         Ratts Tyerell     79   15       none  grey, blue
   eye_color birth_year gender   homeworld        species
3        red      33BBY   <NA>       Naboo          Droid
8        red       <NA>   <NA>    Tatooine          Droid
19     brown     896BBY   male        <NA> Yoda's species
29     brown       8BBY   male       Endor           Ewok
39    orange       <NA>   male   Malastare            Dug
45    yellow       <NA>   male     Vulpter     Vulptereen
72      <NA>       <NA>   male Aleen Minor         Aleena
\end{Soutput}
\end{Schunk}

Para poder ver con más detalle la división en clusters de los datos, se representarán gráficamente los clusters en 
dos gráficas distintas. Para ello harán falta las siguientes librerías.

\begin{Schunk}
\begin{Sinput}
> install.packages("cluster")
> install.packages("factoextra")
> library(cluster)
> library(factoextra)
\end{Sinput}
\end{Schunk}

Ahora que están instalados los paquetes, se pueden representar los resultados anteriores.

\begin{Schunk}
\begin{Sinput}
> fviz_cluster(clasiKmeans,data=medidas)
\end{Sinput}
\end{Schunk}
\includegraphics{memoria4-020}
\begin{Schunk}
\begin{Sinput}
> clusplot(personajesKmeans,clasiKmeans$cluster,color=TRUE,shade=TRUE,labels=2,lines=0)
\end{Sinput}
\end{Schunk}
\includegraphics{memoria4-021}

Una comprobación que se puede realizar para los datos es obtener el número óptimo de clusters para clasificarlos.
Para comprobarlo, se utilizan la medida de la silueta, que indica de forma aproximada cómo de bien se han clasificado
los datos, es decir, si están en los clusters correctos. 

\begin{Schunk}
\begin{Sinput}
> fviz_nbclust(medidas,kmeans,method="silhouette")
\end{Sinput}
\end{Schunk}
\includegraphics{memoria4-022}

Como se puede observar, los mejores números de clusters para clasificar los datos están entre 2 y 3, con 1 cluster 
la clasificación es muy ineficiente y más clusters de 3 son innecesarios.

Como conclusión de esta clasificación de k-means podemos sacar que todos aquellos que quedan en el Cluster 1 comparten que su género
es \textbf{masculino}. Los personajes pertenecientes al Cluster 3 también comparten que son de \textbf{género masculino}, a excepción de R2-D2
y R5-D4 que son droides de pequeña estatura. Sobre el Cluster 3 también cabe destacar que ninguno de los personajes es de Especie \textbf{humano}.
Por último, en el Cluster 2 es donde se acumulan la mayor cantidad de personajes y donde es más dificil es diferenciarlos por atributos, aunque
si se puede observar que \textbf{todos los humanos} (a excepción de Darth Vader) se encuentran en este cluster, rondando los 179cm y 24kg.

\subsection{Clasificación mediante Clustering Jerarquico Aglomerativo}

A continuación realizaremos la clasificación mediante \textbf{Clustering Jerarquico Aglomerativo} en inglés HAC. Mediante este método todos los
personajes irán agrupandose en clusters de pequeño tamaño y poco a poco irán generando clusters mayores hasta lograr un solo cluster que agrupamiento
todos los personajes.
Como hemos aprendido en clase, el primer paso de este algoritmo es calcular la matriz de distancias entre todos los personajes. Para ello
utilizaremos la función \textbf{dist} y el parámetro \textbf{"euclidean"} para calcular las distancias mediante la formula euclidiana:

\begin{Schunk}
\begin{Sinput}
> distancias <- dist(medidas, method = "euclidean")
\end{Sinput}
\end{Schunk}

Una vez hemos calculado las distancias vamos a realizar la clasificación jerarquica. Existen tres métodos diferentes de agrupar en clusters:
\textbf{MIN("single")}, \textbf{MAX("complete")} y \textbf{Group Average("average")}.

\subsubsection{Enlace simple MIN}

Realizaremos la clasificación utilizando la distancia \textbf{mínima} entre puntos a cada cluster:

\begin{Schunk}
\begin{Sinput}
> hcmin <- hclust(distancias, method = "single" )
> plot(hcmin, cex = 0.6, hang = -1, main = "Dendrograma MIN")
\end{Sinput}
\end{Schunk}
\includegraphics{memoria4-024}


\subsubsection{Enlace completo MAX}

Realizaremos la clasificación utilizando la distancia \textbf{máxima} entre puntos a cada cluster:

\begin{Schunk}
\begin{Sinput}
> hcmax <- hclust(distancias, method = "complete" )
> plot(hcmax, cex = 0.6, hang = -1, main = "Dendrograma MAX")
\end{Sinput}
\end{Schunk}
\includegraphics{memoria4-025}

\subsubsection{Enlace promedio Group Average}

Realizaremos la clasificación utilizando la \textbf{media de las distancias} entre puntos a cada cluster:

\begin{Schunk}
\begin{Sinput}
> hcavg <- hclust(distancias, method = "average" )
> plot(hcavg, cex = 0.6, hang = -1, main = "Dendrograma Group Average")
\end{Sinput}
\end{Schunk}
\includegraphics{memoria4-026}

Como se puede observar, todos los dendogramas son muy parecidos entre si, con ligeros cambios debido al método de agrupación escogido
para cada uno.

\subsection{Clasificación mediante DBSCAN}

Por último, realizaremos la clasificación mediante el algoritmo \textbf{DBSCAN}. Este algoritmo agrupará nuestros personajes en clusters
basandose en la densidad, comenzando por una estimación de la distribución de densidad de cada personaje. Utilizaremos los paquetes
\textbf{fpc} y \textbf{dbscan}:

\begin{Schunk}
\begin{Sinput}
> install.packages("dbscan")
> install.packages("fpc")
> library("dbscan")
> library("fpc")
\end{Sinput}
\end{Schunk}

Una vez hemos instalado los paquetes necesarios, calcularemos cual es el valor de \textbf{eps} óptimo para nuestra clasificación DBSCAN.
Para ello, utilizaremos el método k-nearest neighbour, y buscaremos donde exista un cambio drástico en nuestra gráfica:

\begin{Schunk}
\begin{Sinput}
> kNNdistplot(medidas, k=4)
> abline(h=16.5, col="red")
\end{Sinput}
\end{Schunk}
\includegraphics{memoria4-028}

En este ejemplo podemos observar que el cambio se encuentra en el \textbf{valor 16.5}, por lo que ese será nuestro valor para eps. A continuación,
realizaremos la clasificación dbscan y mostraremos una gráfica con los resultados:

\begin{Schunk}
\begin{Sinput}
> set.seed(123)
> clasificacionDBSCAN <- fpc::dbscan(medidas, eps = 16.5, MinPts = 5)
> hullplot(medidas, clasificacionDBSCAN$cluster)
\end{Sinput}
\end{Schunk}
\includegraphics{memoria4-029}

Como podemos observar en el gráfico, el algoritmo nos ha separado los personajes en 2 clusters, un pequeño cluster de color \textbf{verde}
que incluye 6 de estos, y un gran cluster \textbf{rojo} que incluye la gran mayoria, 47. Por último encontramos aquellos puntos negros que
no están ni en el cluster verde ni en el rojo, estos puntos se corresponde con valores \textbf{outlier}.

\end{document}
